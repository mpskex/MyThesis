\chapter{多人人体姿态估计}
\label{cha:related}
本章主要根据网络设计中涉及到领域进行了相关工作的整理与综述。多人姿态估计主要涉及了目标检测、实例分割、单人姿态估计等计算机视觉中常见的相关任务。我们就从对这三个以及多人姿态估计任务的近年研究展开,对多人姿态估计整体的相关工作进行论述。
\section{目标检测}
\label{sec:detect}
目标检测的目标是给出尽可能准确的目标边界框。近年来,由于其突出的鲁棒性、泛化性,基于深度学习的目标检测方法已经成为了研究者们关注的主要方向。在本节,我们主要将按照目标检测发展顺序来总结每个类别下的典型方法。
\subsection{两阶段目标检测方法}
\label{2stagedetector}
两阶段目标检测方法的起始是区域卷积神经网络(R-CNN)\cite{Girshick_2014_CVPR}。R-CNN提出了使用并行的回归分支来得到图中不同位置分类结果,最后组织出一个稠密的分类结果图。通过对结果图的分析能够得到不同类别物体的位置,并
Faster R-CNN
\subsection{单阶段目标检测方法}
\label{subsec:1stagedetector}
YOLO\cite{redmon2016you}的出现标志着单阶段目标检测方法的出现。
Single Shot Detector

\section{实例分割}
\label{sec:insseg}
Mask R-CNN

\section{单人人体姿态估计}
\label{sec:singlepose}
\subsection{基于传统计算机视觉方法的人体姿态估计方法}
\label{subsec:legacypose}
DPM为主的相关工作
\subsection{基于深度学习的人体姿态估计方法}
\label{subsec:deeppose}
基于深度学习的人体姿态估计方法的典型是卷积姿态网络(CPM)\cite{wei2016convolutional}。CPM有两个主要的贡献:隐式地让网络学习人体姿态中不同关键点地信息,而不需要使用任何先验去约束或推断关键点信息;提出了中继监督和多阶段网络设计,并且强调了感受野对姿态估计地重要性。其中中继监督与多阶段的网络设计在之后很多的基于深度学习的姿态估计方法中都有体现。其中堆叠沙漏网络(SHN)\cite{newell2016stacked}就是一个较为典型的例子。SHN提出了沙漏结构,一定程度上融合了不同尺度的特征,让网络能够得到更加精细的关键点估计结果。SHN也一样将沙漏架构堆叠起来,形成一个多阶段的结构,并同样引入中继监督,来避免其过深的网络带来的梯度消失的问题。

由于SHN的特征融合的确较好地提高了模型的准确度,因此之后的工作都大多从特征融合这个方向来改进方法。其中比较典型的就是级联金字塔网络(CPN)\cite{Chen2017Cascaded}。CPN提出如果希望得到更好的关键点结果,那么就必须从不同尺度的特征入手。CPN使用了来自所有尺度的特征,并使用双线性插值上采样的方法一并放大至最底层特征图的大小。这些特征直接被合并在一起,放入优化网络得到最终的特征点热图。这样的方法极大地改善了网络的准确度,然而带来的就是庞大的参数量与计算量。这个方法不论是从训练还是预测角度来讲,都是难以操作的。但是毋庸置疑的是,CPN给我们在解决姿态估计任务提供了一个明确的方向,如果能提出更高效的特征融合方法,那么就有可能在姿态估计任务上获得更好的效果。

\section{多人姿态估计}
\label{sec:multipose}
\subsection{自顶向下的多人姿态估计方法}
\label{subsec:topdownpose}
CPN \& other skeleton
\subsection{自底向上的多人姿态估计方法}
\label{subsec:bottomuppose}
在自底向上的方法中,OpenPose\cite{Cao2016Realtime}是十分典型的方法。
%	TODO: bookmark in chapter 02

\section{本章小结}
综合考量了目标检测,实例分割以及单人人体姿态估计的方法

