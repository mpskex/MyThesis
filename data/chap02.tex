\chapter{多人人体姿态估计}
\label{cha:related}
本章主要根据网络设计中涉及到领域进行了相关工作的整理与综述。多人姿态估计主要涉及了目标检测、实例分割、单人姿态估计等计算机视觉中常见的相关任务。我们就从对这三个以及多人姿态估计任务的近年研究展开,对多人姿态估计整体的相关工作进行论述。
\section{目标检测}
\label{sec:detect}
目标检测的目标是给出尽可能准确的目标边界框。近年来,由于其突出的鲁棒性、泛化性,基于深度学习的目标检测方法已经成为了研究者们关注的主要方向。在本节,我们主要将按照目标检测发展顺序来总结每个类别下的典型方法。
\subsection{两阶段目标检测方法}
\label{2stagedetector}
两阶段目标检测方法
Faster R-CNN
\subsection{单阶段目标检测方法}
\label{subsec:1stagedetector}
YOLO\cite{redmon2016you}的出现标志着单阶段目标检测方法的出现。
Single Shot Detector

\section{实例分割}
\label{sec:insseg}
Mask R-CNN

\section{单人人体姿态估计}
\label{sec:singlepose}
\subsection{基于传统计算机视觉方法的人体姿态估计方法}
\label{subsec:legacypose}
DPM为主的相关工作
\subsection{基于深度学习的人体姿态估计方法}
\label{subsec:deeppose}
CPM为主的相关工作

\section{多人姿态估计}
\label{sec:multipose}
\subsection{自顶向下的多人姿态估计方法}
\label{subsec:topdownpose}
CPN \& other skeleton
\subsection{自底向上的多人姿态估计方法}
\label{subsec:bottomuppose}
openpose

\section{本章小结}
综合考量了目标检测,实例分割以及单人人体姿态估计的方法

