\begin{conclusion}
	
本文提出一种全新的融合人体实例分割信息的多人姿态检测网络结构。该结构将实例分割信息加入到多阶段的优化网络设计中,同时优化分割结果与姿态估计结果。实例分割帮助生成空间注意力引导姿态分支对特定实例进行姿态估计,同时关键点任务中的特征反向辅助分割结果的优化。

为了能够让网络形成考虑遮挡区域的人体划分从而在特征层面上指导关键点估计结果的生成,本文提出了考虑人体遮挡区域的空间注意力来帮助关键点分支关注特定的实例。同时,实例分割也会同时接收来自关键点分支的特征与自生成的注意力来优化当前阶段的实例分割结果。这不光让网络能够提升关键点预测的召回率,同时也让网络获得了对不精确检测框的适应性。

本文通过引入弱监督学习,使自生成的空间注意力学习如何预测被遮挡区域,让联合损失函数间接地作用于注意力转换器从而训练注意力的生成。通过两支任务损失函数的互补,网络生成的注意力能一定程度地考虑被遮挡的人体区域,并在对应区域给出响应。在弱监督学习的约束下,网络能够根据数据学习关键点任务需要关注的区域,并根据实例分割信息约束区域形状的生成。

本文提出的结构与技术能够被推广到实例分割与关键点定位这两个任务以外的其他任务上。首先,引入注意力机制让网络自主学习关注区域可以结合任务中的监督信息,使用数据驱动的方式让网络自适应训练数据集生成对应的关注区域,能够有效提高网络的泛化性;同时本文提出的弱监督学习的训练方法,可以用来训练难以通过显式监督约束的输出的生成。这让网络能够通过联合损失函数的定义监督一个标注信息不完整或不准确的困难任务。

然而本文提出的结构仍然存在一定的缺陷。本文的结构没有充分考虑优化模块中周围注意力的生成。换言之,网络仅仅会自顾自地生成关于一个实例的注意力而不考虑其他实例注意力的生成。在之后的工作中,本文会继续尝试融合全局的注意力信息生成最终的姿态预测结果。

\end{conclusion}