\chapter{绪 论}
\label{cha:intro}

% 课题引言

\section{研究背景}
\label{sec:generalbackground}
人体姿态是计算机视觉领域中一个在现实场景中有极其重大意义的课题方向。一旦掌握了识别人体姿态的能力,计算机可以在各种与人有关的互动中扮演更为重要的角色。近年来,人体姿态估计不光在工业环境应用广泛,在商业上也表现出相当突出的尽头。在自动驾驶中,行人行为的预判是对驾驶决策相当重要,能够降低自动驾驶车辆与行人发生碰撞的可能性;在医疗康复中,姿态估计可以帮助医疗机构监督病人状态,防止出现跌倒的情况;在运动训练中,姿态估计可以被用于动作矫正系统,用来监督运动员的姿态标准性以及检验训练的有效性;在安防方面,行人姿态跟踪与步态识别对于我们追踪与分析潜在犯罪嫌疑人也有一定的帮助;在商业娱乐下,姿态信息也是人机交互的重要接口,在此方面的应用更是数不胜数。因此,寻找一个性能高,应用广的姿态估计方法对于计算机视觉领域是一个志在必得的目标,这也是近年来学界一直关注的焦点课题之一。

目前来讲,基于光学的姿态采集与追踪设备已经是一个非常成熟的技术。然而,所有使用非可见光的姿态采集设备都需要采集者设置多个标定好的特殊摄像机,例如高帧率红外摄像机,在封闭的室内场景中,对被采集者身上的光学追踪点进行定位。由于其捕捉人数有限,环境配置成本较高且需要被采集这佩戴跟踪点的原因,限制其只能应用于计算机渲染动画中。而如果姿态信息能在更加开放的环境被捕捉到,例如一台任意内参\footnote{相机的内参是相机横纵轴的焦距以及光心偏移}、任意外参\footnote{相机的外参是相机的位置姿态,通常是以一个六自由度的矩阵描述。}的,架设在室外场景的单目图像采集设备,不光可以拓宽该技术的应用场景,还可以大大降低该技术的成本,成为一种能够一般企业甚至个人负担起的技术。从2012年的AlexNet\cite{alex2012alexnet}开始兴起的深度学习和卷积神经网络热潮,让这种假设逐渐变成了现实。通过正确的表达,卷积神经网络可以完成在对于任意图像的人体姿态估计。

近年来基于深度学习的人体姿态估计方法层出不穷,从2014年的DeepPose\cite{toshev2014deeppose}到2018年的PersonLab\cite{Papandreou2018PersonLab},可以看到学界对于这个问题的解决思路基本趋于一致:多阶段网络堆叠与中继监督方法结合来设计网络。对于人体姿态的表达也逐渐趋同:使用一组描述人体关键点位置的热图来表示网络最终给出的预测结果。总而言之,现有的人体姿态估计方法都是将姿态信息用关键点的形式表现以后,将回归关键点的任务化为回归关键点的位置概率图的方法。

在多人姿态估计方面,近年来出现的方法可以主要分为两大类:自顶向下,先定位目标位置再预测关键点位置,例如区域多人姿态估计\cite{fang2017rmpe}和自底向上,先预测所有的关键点位置,再使用关键点关系重建单人姿态,例如OpenPose\cite{Cao2016Realtime}。然而这两种方法都有相应的缺点,对于自顶向下的方法而言,一个精确的检测结果是非常重要的:如果检测结果有些许的偏移,那么也许就会漏掉一些目标的部分,这对于后面的关键点回归是一个灾难性的后果。同时,对于自底向上的方法,如何表达一个容易回归,鲁棒的关键点之间的关系也存在一定的难度,因此现有的自底向上都会在一些极端情况中因为结果不够鲁棒而导致重组关键点的结果出现异常\footnote{这些极端情况指,一些躯干相近的预测,如果躯干的朝向相近或边缘模糊,这些方法都会难以给出准确的关系预测,尤其是对于使用向量场表达关系的方法。}。

\section{研究目的}
\label{sec:generalmotivation}
为了解决上文提到的两种困难,本文希望借鉴自顶向下与自底向上方法的优势,设计出一种网络结构,能够一定程度上解决他们各自存在的劣势。这个方法应该对检测结果更加宽容,还能够利用比较鲁棒的线索,将检测到的关键点重组为完整的骨架。

在多人姿态检测中,目标检测被广泛应用,然而我们使用的一般都是检测到的边界框,并没有把目光放在实例分割上。实例分割的目标是将图像中出现的所有可描述的物体都分别勾勒出来。而且与一般的分割任务不同的是,即使两个物体属于同一个类别,实例分割仍然希望能够区分这两个物体。所以,即便在他们相互距离很近,甚至是遮挡的情况下,实例分割仍能够清晰地勾勒出单个人体的轮廓。并且对于监督而言,它相当的鲁棒。然而这里有一个问题,我们如果将实例分割直接用作重组关键点的信息,那么被遮挡的部分是无法被考虑进去的。因为学界在定义分割任务时,就是不考虑被遮挡的部分作为训练标签的。那么我们如何寻找一个对于人体轮廓更加宽松的,并且更鲁邦的表达呢?

% Bookmark here introduce weakly supervised attention here.

定位人体位置。
估计人体姿态。

\section{研究内容}
\label{sec:generalfield}
改进现有人体估计网络结构。
并提出一种可以融合人体实例分割的姿态估计网络结构。

\section{本章小结}
这里完成本章小结