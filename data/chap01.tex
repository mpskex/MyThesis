\chapter{绪 论}
\label{cha:intro}

% 课题引言
本章首先介绍基于单目图像的多人姿态检测与跟踪课题的研究背景及意义。之后,介绍近年相关工作,并分析现有工作存在的问题。在此基础上,给出本文研究内容以及创新点。最后是本文整体结构。

\section{研究背景和意义}
\label{sec:generalbackground}
人体姿态检测与跟踪旨在从传感器数据中捕捉多人姿态信息,在各领域具有广阔的应用前景,其中部分技术在特定场景中已得以应用。例如,在自动驾驶中,对行人姿态的准确检测与跟踪能够降低自动驾驶车辆与行人发生碰撞的可能性;在医疗康复中,姿态信息可以帮助医疗机构监督病人状态,以及时发现患者异常状态;在运动训练中,姿态估计可被用于监督运动员姿态是否标准,以辅助动作矫正;在安防方面,行人姿态检测与跟踪有助于追踪与分析犯罪嫌疑人的行为特征;在商业娱乐方面,姿态检测与跟踪是人机交互的重要手段。

姿态检测与跟踪是计算机视觉领域的重要研究课题。现有多人姿态信息提取方法根据输入数据来源分为基于组合传感器的姿态提取方法与基于单目相机的姿态提取方法。基于组合传感器的姿态采集与追踪设备已经相对成熟,例如VICON商用光学动作捕捉系统。然而,使用组合传感器的姿态捕捉方法大多需要多个采集设备,例如多台已标定高帧率红外摄像机或者集成深度相机。由于此类方法通常要求在被采集者身上粘贴光学追踪点来定位其躯干,导致其应用环境受限,通常被应用于室内场景中。相比之下,基于单目相机的多人姿态估计方法,不受限于室内场景,应用更加广泛。同时,多采用通用图像采集设备,比如带有摄像头的移动设备,或是连接至远程计算中心的网络摄像头,其成本普遍低于光学追踪系统。即使在包含多人姿态估计所需计算设备的前提下,基于单目相机的多人姿态提取方法成本也远低于基于组合传感器的追踪系统的开销。因此,基于单目相机的多人姿态提取方法相比基于组合传感器的姿态提取方法拥有更高的研究价值与应用前景。

然而,基于单目相机的多人姿态提取面临诸多技术难点,其中最难以解决的是多人场景中的互遮挡问题。现有多人姿态估计方法在不同个体躯干相互遮挡的情况下,难以根据可视部分推断不可见部分的信息,从而确定被遮挡部位的姿态。例如,当图像中的特定人肩部被遮挡时,现有算法难以给出其准确的肩部位置。鉴于上述事实,本文在研究现有姿态估计方法的基础上,提出基于实例分割的多人姿态估计方法,从而更好地解决多人姿态估计过程中的互遮挡问题,实现准确的多人姿态检测与跟踪。

\section{相关研究工作}
\label{sec:related_work}
多人姿态估计是多人姿态检测与跟踪方法的核心。基于单目相机的多人姿态估计方法可以分为自顶向下与自底向上两种类型。自顶向下的多人姿态估计方法的主要特点就是在得到人体目标位置的前提下去预测人体姿态;而自底向上方法特点就是先完整的预测人体姿态,再将零散的姿态信息根据特征或其他的表达重组,建立独立完整的人体姿态。

\subsection{自顶向下的多人姿态估计方法}
\label{subsec:topdown}
目前常见的多人姿态估计方法主要都是使用自顶向下的格局设计网络的。算法会先提取人体位置,接下来经过裁剪以后送入姿态估计网络中得到姿态估计。这样的方法可以借助于高性能的单人姿态估计网络和人体检测网络来快速获得满意的效果。然而同时这种特性会带来误差传播的问题:检测结果的误差被传播如姿态估计阶段,导致自顶向下的方法普遍对于准确的目标检测结果依赖很高。近年改善自顶向下的多人姿态估计方法的工作中,Fang等提出的区域多人姿态估计方法\cite{fang2017rmpe}(RMPE)改善了以往自顶向下方法中目标检测不精确结果对于姿态估计的负面影响,通过设计简单的对称的空间变换网络\footnote{对称的空间变换网络,Symmetric Spatial Transformation Network,简称SSTN}依据人体姿态检测结果训练空间变换的参数。该方法一定程度上改善了误差传播问题,但空间变换网络输出仍然会限制空间变换的范围。如果初始检测结果无法得到人的位置,或者偏差过于严重,仍然会导致最后姿态估计的失败。同样地,虽然Chen等的方法\cite{Chen2018Cascaded}也使用了自顶向下的设计,但是该工作仅通过一个更加准确的检测器解决误差传播的问题,并没有通过结合检测与姿态估计任务来改善最终结果。综上所述,现有方法仍没有对误差传播这个问题提出很好的策略来解决不精确检测结果对于姿态估计的影响。

\subsection{自底向上的多人姿态估计方法}
\label{subsec:bottomup}
在使用自底向上结构设计的方法中,同样存在一些结构设计带来的问题。由于该类型方法需要一次性检测所有的关键点,再对这些关键点进行重组。因此,会存在一些在关键点检测阶段中出现的错误位置,被划分进入姿态骨架中。这些被错误检测到的结果可能是非关键点,或是一些不足以构建成为骨架的孤立关键点。同时,与自顶向下仅需要在被提取的兴趣区域上提取关键点不同,自底向上结构需要在完整图像上寻找所有可能的关键点,这对于使用卷积神经网络生成稠密概率图提取姿态的方法而言,会明显增加计算量。近年来的自底向上方法可以按照重组关键点的方法分为两类:使用向量场的与使用分割划分的两种方法。Cao等的工作\cite{Cao2016Realtime}在之前CPM工作的基础上,额外增加一个并行的、堆叠多层大卷积核的分支用来回归部件向量场(PAF, Part Affinity Field)。这个经过多阶段回归的向量场被用来连接不同关键点。每一个关系使用一对稠密向量场来描述对应的热图上每个位置的向量方向。该方法在稠密张量上求解向量与额外的网络分支会增加过多的计算量。并且在使用向量场恢复关键点连接时,算法可能会在一对来自不同人的相同躯干遮挡的情况下,给出错误的连接结果。

另外在使用划分的自底向上多人姿态估计方法中,Newell等的工作\cite{Newell2017Associative}又提出一种新的自底向上的多人姿态估计方法。文章中使用关联编码(Associative Embedding)来让网络自身学习对关键点进行类别标注。然而这种使用网络内部编码的方式定义的标签可解释不强,并且也并不能很精细地区分人与人之间的分割。除此之外,在Gong等的工作中\cite{gong2018instance},提出了可以用于自底向上方法的实例部件分割,结合语意分割与边缘检测来划分人体。在上篇工作的基础上,Liang等使用一种基于精细的部件分割的人体划分方法\cite{liang2019look}。这种方法严格考虑了人体的精细结构,然而精细部件分割的数据标注需要比实例分割标注更多的时间。这就意味着完成该方法所提出的监督方式是需要更多的成本的。这对于一些庞大的数据集而言是很难实现的。

\section{研究内容及创新点}
\label{sec:contribution}
本文提出一种融合人体实例信息的多人姿态估计与分割联合优化网络结构。该结构将实例分割信息融合至多阶段的双任务联合优化过程中,能够同时优化实例分割与姿态估计结果。通过引入实例分割线索,所提出方法更好地解决了多人场景下的互遮挡问题,同时反向辅助了实例分割结果的优化。


本文引入注意力机制有效融合两任务分支中的线索,使得所提出网络能够考虑人体的被遮挡区域,完整准确地估计人体姿态,同时精炼实例分割结果。注意力机制的引入能够使网络从特征层面对人体区域进行划分,使网络专注地提取特定人体的姿态信息。实例分割能够为空间注意力的生成提供线索,然而其真值未能考虑被遮挡的人体区域,导致现有实例分割的标注信息难以用于监督本文注意力模块的学习,使得注意力模块能够同时关注被遮挡部分。因此,本文参考弱监督学习\cite{10.1093/nsr/nwx106},提出一种特殊的监督方法,使用来自实例分割与姿态估计的监督信息宽松地约束空间注意力的生成。


本文的主要贡献包括:
\begin{itemize}
	\item 提出一种全新的融合人体实例分割的多人姿态联合优化网络结构,使用由实例分割生成的空间注意力帮助改善多人姿态估计结果,同时借助关键点检测补全分割预测中缺失的部分;
	\item 使用注意力机制在特征层面对关键点结果进行划分,让网络根据监督信息自主生成所需要的关注区域,对人体姿态在实例层面上进行姿态估计;
	\item 提出了一种宽松的监督方式让网络生成对应的空间注意力,以改善分割信息对不可见身体部分的监督缺失,让空间注意力能够对遮挡区域做出响应。
\end{itemize}

\section{文章结构}
\label{sec:introconclusion}
本文在第二章简要介绍了多人姿态估计的相关技术,包括基于深度卷积神经网络的特征提取、基于深度学习的目标检测以及基于深度学习的人体姿态估计技术。本文在第三章中描述所提出的网络框架,阐明了设计思路并分析了。第四章中本文设计了对应的消融实验以及在COCO 2017数据集上的性能测试以验证本文提出结构的有效性。最终,文章给出了对该工作的总结与展望。