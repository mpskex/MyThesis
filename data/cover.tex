\thusetup{
  %******************************
  % 注意:
  %   1. 配置里面不要出现空行
  %   2. 不需要的配置信息可以删除
  %******************************
  %
  %=====
  % 秘级
  %=====
  secretlevel={秘密},
  secretyear={10},
  %
  %=========
  % 中文信息
  %=========
  ctitle={基于弱监督的特征层级注意力的多人姿态估计},
  cdegree={工学学士},
  cdepartment={信息学部},
  cmajor={信息安全},
  cauthor={刘方瑞},
  csupervisor={马伟副教授},
  studentno={15143103},
  %cassosupervisor={陈文光教授}, % 副指导老师
  %ccosupervisor={某某某教授}, % 联合指导老师
  % 日期自动使用当前时间,若需指定按如下方式修改:
  % cdate={超新星纪元},
  %
  % 博士后专有部分
  catalognumber     = {分类号},  % 可以留空
  udc               = {UDC},  % 可以留空
  id                = {编号},  % 可以留空: id={},
  cfirstdiscipline  = {计算机科学与技术},  % 流动站(一级学科)名称
  cseconddiscipline = {系统结构},  % 专 业(二级学科)名称
  postdoctordate    = {2009 年 7 月——2011 年 7 月},  % 工作完成日期
  postdocstartdate  = {2009 年 7 月 1 日},  % 研究工作起始时间
  postdocenddate    = {2011 年 7 月 1 日},  % 研究工作期满时间
  %
  %=========
  % 英文信息
  %=========
  etitle={Weakly Supervised Feature-level Attention for Instance-aware Multi-Person Pose Estimation},
  % 这块比较复杂,需要分情况讨论:
  % 1. 学术型硕士
  %    edegree:必须为Master of Arts或Master of Science(注意大小写)
  %             “哲学、文学、历史学、法学、教育学、艺术学门类,公共管理学科
  %              填写Master of Arts,其它填写Master of Science”
  %    emajor:“获得一级学科授权的学科填写一级学科名称,其它填写二级学科名称”
  % 2. 专业型硕士
  %    edegree:“填写专业学位英文名称全称”
  %    emajor:“工程硕士填写工程领域,其它专业学位不填写此项”
  % 3. 学术型博士
  %    edegree:Doctor of Philosophy(注意大小写)
  %    emajor:“获得一级学科授权的学科填写一级学科名称,其它填写二级学科名称”
  % 4. 专业型博士
  %    edegree:“填写专业学位英文名称全称”
  %    emajor:不填写此项
  edegree={Bachelor of Engineer},
  emajor={Faculty of Information},
  eauthor={Liu Fangrui},
  esupervisor={Professor Ma Wei},
  %eassosupervisor={Chen Wenguang},
  % 日期自动生成,若需指定按如下方式修改:
  % edate={December, 2005}
  %
  % 关键词用“英文逗号”分割
  ckeywords={计算机视觉, 人体姿态估计, 实例分割, 注意力机制, 弱监督学习},
  ekeywords={Computer Vision, Human Pose Estimation, Instance Segmentation, Attention Mechanism, Weakly Supervised Learning}
}

% 定义中英文摘要和关键字
\begin{cabstract}
  由于在自然图像中的人体存在大量遮挡,因此多人姿态估计是一个相当有挑战的任务。现阶段从单目图像中估计多人姿态的方法可以被分为自顶向下与自底向上两种,然而两种方法都有他们各自的劣势。
  
  本文提出了一个用于多人姿态估计的全新的结构。这个结构借鉴了两种方法的设计思路,并能够融合实例分割的结果,同时优化姿态估计与分割结果。通过在现有的姿态估计分支中引入注意力机制,让网络获得了分辨不同人体的能力,一定程度上解决了遮挡的问题。网络中的注意力使用了弱监督学习的方法,使用关键点与实例分割的两个损失函数完成对其的约束。
  
  通过上述结构,本文提出的方法能够在相比自顶向下和自底向上方法更少计算冗余的条件下,有效的得到准确的姿态估计结果。通过实验我们证明了,该结构能够得到令人接受的结果,还有注意力机制在方法中的有效性。

  本文的创新点主要有:
  \begin{itemize}
    \item 提出了全新的网络结构,用于同时优化实例分割与姿态估计结果;
    \item 引入注意力机制让网络自主学习关注区域;
    \item 对于难以显式监督的注意力,使用弱监督方法隐式限制注意力的形成。
  \end{itemize}

\end{cabstract}

% 如果习惯关键字跟在摘要文字后面,可以用直接命令来设置,如下:
% \ckeywords{计算机视觉, 人体姿态估计, 实例分割, 注意力机制, 弱监督学习}

\begin{eabstract}
Multi-person pose estimation is a challenging task due to various body shapes and occlusion. Current methods to monocular multi-person pose estimation are categorized into top-down and bottom-up methods, while both of these two approaches have their own constraints.

In this work, we propose a novel hybrid architecture for multi-person pose estimation, which benefit from both top-down and bottom-up methods. With an existing global pose extraction branch, we introduce attention mechanism to make the network capable of distinguish by instances. The model is designed refine the result of both segmentation and pose estimation. The attention is trained in weakly supervised learning fashion, being constrained by both two refine loss of instance segmentation and key-points.

With the help of mutual promotion from the proposed architecture, our method is able to generate accurate pose results with less redundant than single top-down or bottom-up approaches. Experiments on benchmark shows that the our model achieves comparable results to other state-of-the-art methods.
\end{eabstract}

% \ekeywords{Computer Vision, Human Pose Estimation, Instance Segmentation, Attention Mechanism, Weakly Supervised Learning}
