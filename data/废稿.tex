\section{研究目的}
\label{sec:generalmotivation}
由于现有算法都难以胜任实时场景的部署,为了算法性能的更好表现,网络还应该简化冗余的网络结构,让计算复杂度与空间复杂度在较小的开销小,尽可能好地完成这个任务。同时为了解决上文提到的两种困难,本文希望借鉴自顶向下与自底向上方法的优势,设计出一种网络结构,能够一定程度上解决他们各自存在的劣势。这个方法应该能够利用比较鲁棒的线索,将检测到的关键点重组为完整的骨架。

随着方法精度的不断上升,很多工作的侧重方向开始向优化计算复杂度靠拢。确实,现在的很多应用场景对使用深度学习的方法寄予了厚望,然而其巨大的计算开销让很多公司难以将其实际用于个人业务。为了降低计算成本,并充分利用好深度学习良好的灵活性,本文希望能通过减少网络计算量来让该算法获得更多地可能性。

为了降低网络计算与存储带来的开销,方法需要注意减少网络结构的冗余。对于自顶向下的方法而言,像区域姿态网络\cite{fang2017rmpe},需要一个独立的目标检测网络来完成人体边界框的提取,同时在姿态提取阶段,一个相似的深度网络也是必须的:网络需要获得更深的结构来获得更好地结果。那么这一部分相似的深度网络结构是冗余的,完全可以通过简单线性组合来重新组织各个层次的特征来获得等价的效果。而且对于两个相似的任务,就例如本文中希望利用的实例分割与最终目标多人姿态估计,共享参数层可能会获得更好的效果,因此本文也希望能够在网络上利用共享的特征提取方法来达成这个目标。

在多人姿态检测中,目标检测被广泛应用,然而在这些姿态检测方法中使用的信息一般都是检测到的边界框,并没有把目光放在实例分割上。实例分割与语义分割任务不同,其目标是将图像中出现的所有可描述的物体都分别勾勒出来,也就是说,它能够将每个目标的轮廓回归出来并给出分类。所以,即便在他们相互距离很近,甚至是遮挡的情况下,实例分割仍能够清晰地勾勒出单个人体的轮廓。同时需要提出的一点就是,实例分割与姿态估计是一对相关的任务:实例分割如果直接作用于多人姿态估计结果上可以增强姿态估计的结果;同样地,根据多人姿态估计的结果可以重建躯干的连接,这些连接是可以重新还原一个简单粗略的实例分割结果的。那么在这样一个前提下,网络可以尝试融合实例分割与姿态估计以让其互相促进,获得比仅使用边界框信息的方法更好的结果。


近年来,学界对于迁移学习的研究也越来越深入,已经有工作证明了在训练自己网络的时候使用在较大数据集的分类任务下预训练好参数作为初始状态会获得更好的效果\cite{mishkin2015all}。并且,在Misra, Ishan等人\cite{misra2016cross}的工作中,也提出了对于相近网络中使用共享的参数会给网络带来额外的性能提升。在现有的多人姿态检测方法中,尤其是自顶向下的方法,会使用人体检测网络与姿态提取网络。而这些网络都依赖深层卷积网络来获得足够的感受野,所以不论是监测网络还是姿态网络,都需要一个相似甚至相同的的结构,使用预训练好的参数进行迁移训练,比如在ImageNet\cite{deng2009imagenet}下预训练好VGG-19\cite{simonyan2014very}或者ResNet-101\cite{He2015Deep}。一个很自然的思路就是通过使用共享的特征提取网络来以预训练好的参数作为初始训练一个多任务的网络,同时能够完成实例分割与姿态估计。这样的方法自然就是能够减少网络计算冗余,并且还会获得一定的性能提升。

正如章节\ref{sec:generalmotivation}所讲,本文希望融合实例分割的结果来完成对不同人的区域的区分。不论是语义分割还是实例分割任务,从他们的监督来讲,它们的真值都具有相当强的语义性,这也就意味着系统具有相当的鲁棒性。然而正因为如此,如果将实例分割直接用作重组关键点的信息,那么在真值被遮挡的部分是无法被考虑进去的。那么如何寻找一个对于人体轮廓更加宽松的,并且更鲁棒的表达呢?
% weakly supervised attention
本文借鉴了弱监督的思想,结合注意力机制的概念让网络结合实例分割自主学习出需要学习的区域。由于过强的语义信息导致受到监督的网络对于遮挡过于敏感,所以网络需要一些非直接的,可学习的方法来解决这个问题。本方法希望使用了弱监督的注意力来约束多人姿态的估计结果,从而达到改善原有多人姿态估计中出现的遮挡问题。

弱监督学习是是指在强监督信息不完整或不准确的情况下,使用较宽松的监督信息或结构约束并训练网络的方法\cite{10.1093/nsr/nwx106}。弱监督学习是近几年被提出的。由于标注数据量逐渐增大,数据标注质量良莠不齐,导致很多时候模型因为一些未经过良好标记的数据而在一些任务上难以收敛,甚至于直接失效。同时,一些有特殊需求的任务也需要额外的标注信息去训练。如果使用强监督学习就势必要花费重金去补全所需的标注数据。与其他使用弱监督信息的方法相似,本文使用弱监督的情景也是在实例分割无法满足监督所需要的注意力的要求的情况下引入了弱监督学习,并且设计了合适的结构和与之对应的强监督信息间接地约束注意力的生成。

因此,本文提出了全新的结构,旨在使用实例分割的的信息生成一个空间注意力,用来帮助姿态估计分支得到更加鲁棒的结果。同时这个结构也将姿态估计中多阶段优化的结构引入了实例分割任务中,让分割任务多阶段地与姿态信息融合。这种结构中生成的注意力能够较为有效地解决姿态估计问题中的遮挡问题,并且不会在网络中增加过大的额外计算复杂度。


\chapter{多人人体姿态估计}
\label{cha:related}
本章主要根据网络设计中涉及到领域进行了相关工作的整理与综述。多人姿态估计主要涉及了目标检测、实例分割、单人姿态估计等计算机视觉中常见的相关任务。本文就从对这三个以及多人姿态估计任务的近年研究展开,对多人姿态估计整体的相关工作进行论述。

从2012年的AlexNet\cite{alex2012alexnet}开始兴起的深度学习和卷积神经网络热潮,让这种假设逐渐变成了现实。卷积神经网络是一种深度神经网络,通过多层堆叠可学习参数的卷积核对数字图像进行卷积操作,从而在顶端,也就是网络最终层生成稠密的特征图的过程。卷积神经网络可以很好的适应数据,不仅在大型数据集的分类任务取得了相当优秀的结果,还可以完成像目标检测、姿态估计等等的复杂任务。卷积神经网络可以学习数据集中不同的特征,并能够通过对这些特征的空间关系进行建模以使损失函数的输出靠近最小值。令人欣喜的是,卷积神经网络在仅给出图像标签作为训练数据的情况下,仍然可以在网络中间生成的特征中找到接近人为定义用于检测角与边的卷积核,并在更深层的网络中输出包含更多语意的特征\cite{yosinski2015understanding}。这样的特性让卷积神经网络,也就是CNN获得了相当强的拟合能力和泛化能力,在同样的数据集中训练会有更广的应用。

在多人姿态估计方面,近年来出现的方法可以主要分为两大类:自顶向下,先定位目标位置再预测关键点位置,例如区域多人姿态估计\cite{fang2017rmpe}和自底向上,先预测所有的关键点位置,再使用关键点关系重建单人姿态,例如OpenPose\cite{Cao2016Realtime}。然而这两种方法都有相应的缺点,对于自顶向下的方法而言,一个精确的检测结果是非常重要的:如果检测结果有些许的偏移,那么也许就会漏掉一些目标的部分,这对于后面的关键点回归是一个灾难性的后果。同时,对于自底向上的方法,如何表达一个容易回归的关键点之间的关系也存在一定的难度,因此现有的自底向上都会在一些极端情况中因为结果不够鲁棒而导致重组关键点的结果出现异常\footnote{这些极端情况指,一些躯干相近的预测,如果躯干的朝向相近或边缘模糊,这些自底向上方法都会难以给出准确的关系预测,尤其是对于使用向量场表达关系的方法。}。本文会在\ref{sec:poseestimation}中深入探讨两种类型方法的优势以及面临的挑战。

\section{目标检测}
\label{sec:detect}
目标检测的目标是给出尽可能准确的目标边界框。近年来,由于卷积神经网络突出的鲁棒性、泛化性,基于深度学习的目标检测方法已经成为了研究者们关注的主要方向。在本节,文章主要将按照目标检测发展顺序来总结每个类别下的典型方法。
\subsection{两阶段目标检测方法}
\label{2stagedetector}
两阶段目标检测方法的起始是区域卷积神经网络(R-CNN)\cite{Girshick_2014_CVPR}。R-CNN提出了使用并行的卷积神经网络分支来得到图中不同位置分类结果,最后组织出一个稠密的分类结果图,通过对结果图的分析能够得到不同类别物体的位置。最后经过后处理,去掉每个类别置信度底的边界框并使用非极大值抑制去掉高度重合的边界框,就可以得到准确的检测结果。虽然R-CNN并没有完全给出一个端到端的方法,但是这个工作引出了后面常见的两阶段目标检测方法。

Girshick R.等人在R-CNN的设计中,使用了滑动窗口对图像的每个区域进行分类,并且这些滑动窗口的数目过于庞大。为了减少区域的数量,Faster R-CNN\cite{Ren2015Faster}提出了使用区域建议网络来给为后面的分类分支提供较精确的候选框,供兴趣区域池化层裁剪特征,以便得到最终回归的检测框。两阶段目标检测方法准确度相对更高一些,由于区域建议网络已经给出了粗略的位置信息,并较为贪婪地选取不同尺寸、不同长宽比的候选框,所以其在召回率上会有一定的优势。

\subsection{单阶段目标检测方法}
\label{subsec:1stagedetector}
Redmon J.\cite{redmon2016you}等人的工作,也就是YOLO的出现标志着单阶段目标检测方法的出现。网络主要的思路就是使用全卷积的方式直接对于每个位置进行区域位置的分类与检测框回归。这样的可以省去两阶段的中冗余的区域建议网络的计算以及可能会损失精度的池化层。在速度上,但阶段目标检测也显示出了相当的优势。

为了改善YOLO中出现的漏检问题,Liu W.等人提出了SSD\cite{liu2016ssd},也就是一种单次预测多边界框的检测器。SSD通过从不同感受野尺寸中回归不同大小的边界框来弥补单阶段检测方法中出现的漏检问题。因为对于较大物体的检测,网络需要更大的感受野区域去寻找物体局部特征在物体全局的关系;另外小物体检测中需要较为精细的候选框密度,也就是用来做检测的特征图的尺寸要尽可能大才能保证小物体的检测召回率维持在可接受的范围内。

\subsection{实例分割}
\label{subsec:insseg}
实际上实例分割是目标检测的一个延伸任务。最直观的完成实例分割的方法就是在完成目标检测后使用池化层,将特征输入进分割子网中,求得最终的实例分割结果。正如Kaiming H.等人\cite{He2017Mask}的工作中提到的,使用简单的分割子网就能够比较好地完成实例分割的任务。目前不借助检测结果得到实例分割的方法并不流行,因为寻找同类中不同物体的分割边界对于全卷积网络而言还是有挑战性的,除非能够寻找一种合适的表达来描述不同物体的标记。

\section{注意力机制与目标检测}
\label{sec:attenandobjdet}
注意力机制不光可以被应用于自然语言处理领域,也可以被应用在计算机视觉领域,比如Mnih V.等人的工作\cite{mnih2014recurrent}。该方法使用了循环神经网络的结构,动态地生成下一步网络需要关注的对应区域的注意力,让网络获得每次关注特定区域的能力。

在Vaswani A.等人的工作\cite{vaswani2017attention}之后,非循环网络结构下的自生成软注意力也越来越多地被应用在不同领域中。近年来,研究者把注意力机制应用在目标检测任务上,比如Fei W.等人的工作\cite{wang2017residual},使用了并行的分支生成注意力\footnote{Fei W.等人的工作提出了三种注意力的形式,分别是空间注意力、通道注意力以及混合注意力。其中混合注意力是前两者的混合形式。},并与特征融合从而优化分类的结果。

\section{姿态估计}
\label{sec:poseestimation}
姿态估计任务一直在计算机视觉领域拥有较高的关注度。尤其是在神经网络出现之后,姿态估计问题终于有了较为稳定且性能良好的解决方法。在本节中,文章将从单人姿态估计与多人姿态估计两个方向展开,总结近年来计算机视觉领域在姿态估计任务上的相关工作。
\subsection{单人人体姿态估计}
\label{subsec:singlepose}
在单人姿态估计方法中,传统计算机视觉的方法虽然相对而言时间较为久远,但是仍然能提供比较好的解决思路,因此在这里本文也在本节总结其对姿态估计的贡献。近年来基于深度学习的方法如雨后春笋般出现,我们也挑选了其中在单人人体姿态估计中较为典型的几个工作作为综述的内容。
\subsubsection{基于传统计算机视觉方法的人体姿态估计方法}
\label{subsubsec:legacypose}
基于传统计算机视觉方法的人体姿态估计中最有代表性的就是DPM\cite{felzenszwalb2010object},也就是基于混合多尺度的可变形部件模型。该工作中使用的图结构依赖一系列部件以及部件间的位置关系来表示目标。每个部件描述目标的一个局部属性,通过部件间的弹簧连接表示模型的可变形配置。网络使用高斯金字塔提取图像中的特征,以这些特征作为依据使用隐式支持向量机\footnote{隐式支持向量机,Latent Support Vector Machine}回归最终人体不同部件的位置。这个工作的典型特征,也是一般多数基于传统计算机视觉方法的人体姿态估计使用的技巧,就是利用图模型给人体姿态建立先验,使用这一部分先验来帮助算法得到人体姿态估计结果。
\subsubsection{基于深度学习的人体姿态估计方法}
\label{subsubsec:deeppose}
基于深度学习的人体姿态估计方法的典型是卷积姿态网络(CPM)\cite{wei2016convolutional}。CPM有两个主要的贡献:隐式地让网络学习人体姿态中不同关键点地信息,而不需要使用任何先验去约束或推断关键点信息;提出了中继监督和多阶段网络设计,并且强调了感受野对姿态估计地重要性。其中中继监督与多阶段的网络设计在之后很多的基于深度学习的姿态估计方法中都有体现。其中堆叠沙漏网络(SHN)\cite{newell2016stacked}就是一个较为典型的例子。SHN提出了沙漏结构,一定程度上融合了不同尺度的特征,让网络能够得到更加精细的关键点估计结果。SHN也一样将沙漏架构堆叠起来,形成一个多阶段的结构,并同样引入中继监督,来避免其过深的网络带来的梯度消失的问题。

由于SHN的特征融合的确较好地提高了模型的准确度,因此之后的工作都大多从特征融合这个方向来改进方法。其中比较典型的就是级联金字塔网络(CPN)\cite{Chen2017Cascaded}。CPN提出如果希望得到更好的关键点结果,那么就必须从不同尺度的特征入手。CPN使用了来自所有尺度的特征,并使用双线性插值上采样的方法一并放大至最底层特征图的大小。这些特征直接被合并在一起,放入优化网络得到最终的特征点热图。这样的方法极大地改善了网络的准确度,然而带来的就是庞大的参数量与计算量。这个方法不论是从训练还是预测角度来讲,都是难以操作的。但是毋庸置疑的是,CPN在解决姿态估计任务提供了一个明确的方向,如果能提出更高效的特征融合方法,那么就有可能在姿态估计任务上获得更好的效果。

\subsection{多人姿态估计}
\label{subsubsec:multipose}
多人姿态估计方法可以分为自顶向下与自底向上两种类型。自顶向下的多人姿态估计方法的主要特点就是在得到人体目标位置的前提下去预测人体姿态;而自底向上的方法特点就是先完整的预测人体姿态,再讲零散的姿态信息根据特征或其他的表达重组,建立独立完整的人体姿态。
\subsubsection{自顶向下的多人姿态估计方法}
\label{subsubsec:topdownpose}
目前常见的多人姿态估计方法主要都是使用自顶向下的格局设计网络的,也就是说,算法会先提取人体位置,接下来经过裁剪以后送入姿态估计网络中得到姿态估计。这样的方法可以借助于高性能的单人姿态估计网络和人体检测网络来快速获得满意的效果。在近年改善自顶向下的多人姿态估计方法的工作中,F. Hao-Shu\cite{fang2017rmpe}等人提出的区域多人姿态估计方法(RMPE)改善了以往自顶向下方法中目标检测不精确结果对于姿态估计的负面影响,通过设计简单的对称的空间变换网络\footnote{对称的空间变换网络,Symmetric Spatial Transformation Network,简称SSTN}来动态生成空间变换的参数,也就是人体检测结果来优化人体边界框与人体姿态估计的结果。虽然这种改进仍然以来较为准确的人体目标检测结果,但这一工作还是给原来看似简单的自顶向下多人姿态估计方法提供了新的改善思路。

\subsubsection{自底向上的多人姿态估计方法}
\label{subsubsec:bottomuppose}
在自底向上的方法中,OpenPose\cite{Cao2016Realtime}是十分典型的方法。OpenPose在之前CPM工作的基础上,额外增加了一个类似的分支用来回归向量场。这个向量场用来连接不同关键点。每一个关系使用一对稠密张量来描述对应的热图上每个位置的向量方向。虽然网络在速度上要优于一些自顶向下的方法,但在稠密张量上求解向量与额外的网络分支仍然会增加过多的计算量。并且在使用向量场恢复关键点连接时,算法可能会在一对来自不同人的相同躯干遮挡的情况下,给出错误的连接结果。但是总体而言整个方法为我们用卷积网络完成自底向上的方法给出了很好的引导。

%	TODO: bookmark in chapter 02
另外除了回归向量寻找关键点之间的关系以外,Newell A.\cite{Newell2017Associative}等人的工作又提出了一种新的自底向上的多人姿态估计方法。文章中提出了使用关联编码(Associative Embedding)来让网络自身学习给关键点打标签。网络通过生成一组编码来标记关键点所属的人的标签。这给设计自底向上的多人姿态估计方法又提供了新的思路。


\section{本章小结}
本章总结了目标检测的现代算法,简单分析了两阶段与单阶段目标检测的特性,并且介绍了实例分割\ref{sec:detect}。在之后的一节中\ref{sec:attenandobjdet}本文简单介绍了注意力在目标检测的应用,以及这些方法使用注意力机制的主要思路。在最后一节\ref{sec:poseestimation}本文总结了现有的姿态估计方法以及他们对于该任务的贡献。
