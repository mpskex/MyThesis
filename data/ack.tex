% 如果使用声明扫描页,将可选参数指定为扫描后的 PDF 文件名,例如:
% \begin{acknowledgement}[scan-statement.pdf]
\begin{acknowledgement}
衷心感谢导师马伟教授对本人的精心指导。老师在完成毕设中提供了相当多的帮助与支持。如果没有马伟老师的辅导,本人一定没有办法顺利完成本人的毕业设计。马老师对于本人的毕业论文非常认真。不论是实验设计上或是文章结构上,老师都会倾心给出建议。同时老师也非常重视学生的看法,每次的课题讨论,老师都会结合同学的想法给出建议。同时本人也非常感激马伟老师对我的器重与支持。老师并没有因为本人是从人文学院转系来而拒绝我的申请,而是给了相当多设备上和科研上的支持。对这些来自老师的帮助本人仅用一次致谢难以平复心情,所以再次向老师致以最衷心的感谢,感谢老师对我的付出以及培养。

同时在中国科学院自动化所进行三个月的实习期间,承蒙徐士彪副教授热心指导与帮助。徐老师在实习结束后仍乐于解答本人的困惑,这让本人在接下来的工作中避免了很多不必要的试错,不胜感激。徐老师在答疑解惑的过程中,也教会了许多科研上的经验,这让我受益匪浅。

对以上两位老师谢意难以言表,故作词:器重之恩难相报,桃李之后定相告。

马伟老师实验室的各位师兄师姐们也给本人提供了相当多的帮助。我十分感谢李曈师兄与郑玛娜师姐在他们实验空余时间将GPU借给本人使用。这让我在进度上有了很明显的推进。除此以外我还要感谢李曈师兄,李鹏师兄,龚超凡师兄还有马文广师兄在休息之余给本人提出的很多有用的建议与意见,让我学习了很多前沿领域的知识与技巧。

同时还要感谢本人父母在完成论文时给予的坚实支持。情绪上的稳定与实验进度有着相当大的关系。父母在本人顾及不到之处给了不可缺少的支持,让本人平稳地走过了这相当忙碌的大学四年。所以在论文中必须要感谢父母对本人的理解与关怀,尤其在这大学最后一年中让我专心完成学业。

最后要感谢本人的挚友们。相识的机缘也许不同,但友人们在本人迷茫时给予的关怀是相同的。虽然道路不一样,但是只要一直努力下去,就一定会看到未来。这让我想起了一段话:

\begin{minipage}{0.8\textwidth}
{\fangsong
\hskip2cm没错,我们至今为止所做的一切都不是徒劳的。

\hskip2cm只要我们不止步,前方就会有路....

\hskip2cm所以,不要停下来...}
\end{minipage}
\vskip20pt

本人也会在毕业之后,继续带着各位给本人的祝福与期望,继续努力下去的。

  
\end{acknowledgement}
