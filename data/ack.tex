% 如果使用声明扫描页,将可选参数指定为扫描后的 PDF 文件名,例如:
% \begin{acknowledgement}[scan-statement.pdf]
\begin{acknowledgement}
衷心感谢导师马伟老师对我的精心指导。老师在完成毕设中提供了相当多的帮助与支持。如果没有马伟老师的辅导,我一定没有办法顺利完成我的毕业设计。马老师对于我的毕业论文非常认真。不论是实验设计上或是文章结构上,老师都会倾心给出建议。同时老师非常重视学生的看法,每次讨论中,老师基于我的个人想法都会给出建议。我非常感激马伟老师对我的器重与支持。老师并没有因为我是从人文学院转系来而拒绝我的申请,而是给了相当多设备上和科研上的支持。对这些来自老师的帮助我仅用一次致谢难以平复心情,所以再次向老师致以最衷心的感谢,感谢老师对我的付出以及培养。

同时在中国科学院自动化所进行三个月的实习期间,承蒙徐士彪副老师热心指导与帮助。徐老师在实习结束后仍乐于解答我的困惑,这让我在接下来的工作中避免了很多不必要的试错,不胜感激。徐老师在答疑解惑的过程中,也教会我许多科研上的经验,这让我受益匪浅。

对以上两位老师谢意难以言表,故作词:器重之恩难相报,桃李之后定相告。

马伟老师实验室的各位师兄师姐们也给我提供了相当多的帮助。我十分感谢李曈师兄与郑玛娜师姐在他们实验空余时间将宝贵的GPU资源分享给我。这让我在进度上有了明显的推进。除此以外我还要感谢之前提到的李曈师兄,李鹏师兄,龚超凡师兄还有马文广师兄在休息之余为我提出的诸多建议与意见,以及慷慨分享的论文和书籍,让我学习了很多不同课题前沿领域的知识与技巧,开拓了我解决问题的思路。

同时还要感谢我父母在完成论文时给予的坚实支持。情绪上的稳定与实验进度有着相当大的关系。父母在我顾及不到之处给予了不可缺少的支持,让我平稳地走过了这相当忙碌的大学四年。我要对在这大学四年中让我专心完成学业的父母说一声谢谢。

最后要感谢我的挚友们。相识的机缘也许不同,但友人们在我迷茫时给予的关怀是相同的。虽然道路不一样,但是只要一直努力下去,就一定会看到未来。这让我想起了一段话:

\begin{minipage}{0.8\textwidth}
{\fangsong
\hskip2cm没错,我们至今为止所做的一切都不是徒劳的。

\hskip2cm只要我们不止步,前方就会有路....

\hskip2cm所以,不要停下来...}
\end{minipage}
\vskip20pt

我也会在毕业之后,带着各位给我的祝福与期望,继续努力下去的。

  
\end{acknowledgement}
